\section{Chapter-3-Multiplexing}
The socket number is assigned a port number by the Protocol. One process may deal with one or more numbr of sockets. The 
port number is a \textcolor{blue}{16-bit} number. The port number is used to pass on the \textbf{segment} to the attached process.
POrt number 21 is used for FTP, 23 for Telnet, 25 for SMTP, 80 for HTTP, 110 for POP3, 143 for IMAP, 443 for HTTPS, 993 for IMAPS, 995.
\\
A \textcolor{red}{UDP}  socket is uniquely identified by the destination port number and the destination IP address.

\section{Chapter -4}
Head of line blocking is when the head of line is waiting for a packet from another port to go to the same designated output port,
then the whole line is blocked.
\\
Output port queueing then the arrival rate may become two times the transmission rate, the rate of arrival is higher, this causes queueing, 
if this happens continuously then the datagrams, will start getting dropped. Teh questions is what \textcolor{blue}{Drop policy do I use?}
\\
Datagrams can be lost due to congestion and lack of buffers, think the video traffic is more important than the file traffic, some packets are more
critical than the others.
\textit{switch-fabric -> buffer queueing -> Scheduling-Discipline}, What if the organization is known and then they are given more priority !, this is 
no longer a \textcolor{blue}{fair world}!, that was not what we wanted in the first place,in general te inout port an output port bandwidths are usually a little bit same
\subsection{How much buffering?}
The average buffering time = $RTT \times C$ where $C$ is the capacity of the link. You can't make a large one TB cache because most of the time
the the cache will be \textcolor{red}{tremendously expensive}, when searches are to be conducted.
\\
Recent recommendations state that there may be many TCP senders for information adn sat one million packets are beign sent forma particular group of users
$G_1$ . The group $G_2$ then will send packets that will be waiting for the forst one million packets to be sent, this is the problem,
so \textbf{small buffers are sweet}
\\ Buffer management also performs \textcolor{red}{Marking} of the data in order to signal Network congestion.

\subsection{Packet Scheduling}
There are several techniques
\begin{enumerate}
\item  FCFS - trivial
\item Priority - Here there is a classification into two queues one that is high priority and low priority, there is \textcolor{green}{Starving of low priority packets}
\item Round Robin - Here its an equal world, serves all the queues equally, but the problem is that the high priority packets are not served first.
\item Weighted Fair queueing - \colorbox{BurntOrange}{This is the best one} there is a weighted value of each datagram that is chosen, the average rate - \textcolor{red}{$\frac{w_i}{\sum w_i} R$}
\end{enumerate}
\subsubsection{Network Neutrality}
\textcolor{blue}{Technical} : How should an ISP share it's resources, through the mechanisms
\textcolor{blue}{Social and economic} principles need to be upheld as well, freedom of speech and enncouraging innovation and more.
\\
\lineo
\section{The Internet Protocol}

\subsection{Network Layer}
32 bits in width.
The datagram format as the type fo service the \textcolor{blue}{ECN or diffserv} and the \textcolor{red}{TTL is the maximum number of hops that can be makde by a packet}
\colorbox{BurntOrange}{TTL is decreased per hop}
there is also a flag that will show what the upper layers is UDP or TCP.
There is also a fragmentation check that hels use break the packets into smaller pieces.
overhead is 20(TCP) + 20(IP) = 40 bytes plus IP information
\\
\lineo
\subsection{IP Addressing}
The IP addressing uses sub-netting, i.e the router will have input ports that
take in say \textcolor{blue}{223.1.1.4} and then automatically the inner ones like \textcolor{blue}{223.1.1.2} are connected
\subsection{Interfaces} The interfaces are the face between the host router and the router itself.
Eg: IN the hostels, the order of connnections is like -> \textit{Access point -> Router -> switch}
There is an interface between any two devices that are connected, eg: ethernet interface or wifi station interface.
\\
\textbf{Subnet}: Devices that can physically reach other without passing through an intervening router.
\textbf{IP addresses have structure}: devices in the same subset have common higher order bits
the \textcolor{red}{hostpart is the remaining lower order bits},the subnet makes being 24 means that the first 24 bits are constant
and will not change. The subnet \textcolor{BurntOrange}{may have 23 bits and in this case the subnet will only have even numbers} in the second last octet because of this,
since the last bit of the second last octet, is given away to the host part so that it gets more flexibility.

\subsection{DHCP}
Goal: host dynamically obtains the IP address from the network server when it "joins the network" 
can renew its lease on address in use. allows the reuse of addresses, support for mobile addresses that join and leave the network. 

\subsection{DHCP client-server scenario}
\begin{enumerate}
\item Do not manually configure IP address
\item Is there a DHCP server out there
\item 67 is that standard port for the DHCP communications
\end{enumerate}

\colorbox{Yellow}{\textcolor{blue}{Routers use an inbuilt DHCP server}}

\section{NAT}
The NAT uses the translation of global IP to the private IP, 
then the whole IP address is changed by the NAT
NAT maintains a table of WAN side address and a LAN side Address.
replaces the source IP and source port with the public ones when it is going from inside to outside.
