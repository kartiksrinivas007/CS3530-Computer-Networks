\section{Implementation Details}
Whatever is to be explained is already present inside the \textit{\textcolor{red}{README.md}} file
There are two ways to run the program If you run $\textit{\textcolor{red}{client.py }} $You will have to enter the requests manually This is an implementation of \textcolor{blue}{persistent HTTP} . 
The other way is to use the \textcolor{blue}{impersistent HTTP} method
and just run the shell script $\textit{\textcolor{red}{run\_tests.sh }}$

\subsection{Internal Implementation}
Inorder to send Multiple requests over a connection, there must be something sent to the server that indicates that the connection needs to be closed
This \textcolor{blue}{END} request is sent internally to signify that the connection has ended from the cleint side. This allows graceful termination on
the server side. You can see that Wireshark Reecognizes the normal Requests as HTTP requests as well (See the Pcap files).
The parsing is done by using the python \textit{\textcolor{red}{string.split()}} function.

\subsection{Error Handling}
There has been usage of Error handling using try-except blocks in python. Appropriate Error codes are also sent back to the client.
Here are the Error codes:-
\begin{enumerate}
    \item \textcolor{blue}{HTTP/1.1 200 OK} 
    \item \textcolor{red}{HTTP/1.1 404 Key Not found}
    \item \textcolor{red}{HTTP/1.1 400 Bad Request}

\end{enumerate}


The last one is for the case where the request cannot be parsed. You can test these by running the python file barebones, i.e only run 
\textit{\textcolor{blue}{python client.py}} on the xterm window for node h1\\
\