\section{Implementation Details}
Whatever is to be explained is already present inside the \textit{\textcolor{red}{README.md}} file
There are two ways to run the program If you run $\textit{\textcolor{red}{client.py }} $You will have to enter the requests manually This is an implementation of \textcolor{blue}{persistent HTTP} . 
The other way is to use the \textcolor{blue}{impersistent HTTP} method
and just run the shell script $\textit{\textcolor{red}{run\_tests.sh }}$

\subsection{Internal Implementation}
Inorder to send Multiple requests over a connection, there must be something sent to the server that indicates that the connection needs to be closed
This \textcolor{blue}{END} request is sent internally to signify that the connection has ended from the cleint side. This allows graceful termination on
the server side. You can see that Wireshark Reecognizes the normal Requests as HTTP requests as well (See the Pcap files).
The parsing is done by using the python \textit{\textcolor{red}{string.split()}} function.

\section{Basic Topology Analysis}

\subsection{File Locations}
The \textit{Pcap} files for the \textcolor{blue}{GET PUT and DELETE} requests is under \textit{\textcolor{red}{Real-pcaps/H1-trace-Get-Put-Delete.pcap}}
and the \textit{Pcap} files for the Get request time analysis are under the folder \textit{\textcolor{red}{GET-request-Analysis}}
\subsection{GET - Request Time analysis}
All times have been calculated using \textit{\textcolor{red}{Wireshark}} Timestamps on the client side.
(\textbf{quick trick}: use keystrokes control + T to get time differences quickly )
The GET Requests use \textcolor{blue}{Impersistent HTTP} Method
\\

\begin{center}
    \begin{tabular}{c | c | c | c}

        Key & Request 1 & Request 2 & Request 3 \\
        \hline
        key1 & 0.0009 & 0.0011 & 0.00087 \\
        \hline
        key2 &  0.0014 & 0.0008 &  0.0006\\ 
        \hline
        key3 &  0.0014 & 0.001 & 0.0007 \\
        \hline
        key4 &  0.0009& 0.001 & 0.0006\\
        \hline
        key5 &  0.001 &  0.002 & 0.0007\\
        \hline
        key6 &  0.0009 & 0.0009 & 0.0012\\
        \hline
        \textcolor{blue}{Average} & 0.0010833 & 0.001333 & 0.00076\\
        \hline 
    \end{tabular}
\end{center}
