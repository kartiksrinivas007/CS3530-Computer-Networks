\section{Star Topology Analysis }


\subsection{File Locations}
The \textit{Pcap} files for the \textcolor{blue}{GET} request for key-1 for all three interfaces \textit{\textcolor{orange}{s1-eth1, s1-eth2, s1-eth3 }}is under \textit{\textcolor{red}{Real-pcaps/H1.pcap, H2.pcap, H3.pcap}}
and the \textit{Pcap} files for the Get request time analysis are under the folder \textit{\textcolor{red}{GET-request-Analysis}}
\subsection{GET - Request Time analysis}
The GET Requests use \textcolor{blue}{Impersistent HTTP} Method

\subsection{GET Request Time analysis}
    This is done through \textcolor{blue}{Impersistent HTTP} Requests. \\ The files that contain the pcap traces for the numbers 
    are \textit{\textcolor{red}{First-Time.pcap, Second-Time.pcap, Third-Time.pcap}}
\begin{center}

\begin{tabular}{c |c |c |c }
    Key & Request 1 & Request 2 & Request 3 \\
    \hline
    key1 & 0.00185 & 0.002 & 0.0081 \\
    \hline
    key2 & 0.004 &0.006  & 0.00075 \\
    \hline
    key3 & 0.0042 & 0.001 & 0.0005 \\
    \hline

    key4 & 0.004& 0.001 & 0.0008 \\
    \hline

    key5 & 0.005& 0.004 & 0.0009 \\
    \hline 
    key6 & 0.005232 & 0.0008 & 0.0015 \\
    \hline
    \textcolor{blue}{Average Time} & 0.004047 & 0.0001566 & 0.00087667 \\
    \hline
\end{tabular}
\end{center}
\subsection{Reasons for Results}
As you can see the average time goes down after running the same \textcolor{blue}{GET} requests for the second and third time. This is because
the cache server \textbf{Stores the key's that are requested recently}. In the beginning only \textcolor{red}{key1} is inside the cache. Hence \textbf{initally the time for 
key1 is low}, and that of the other keys is high. But once all the requests have been made for the first time, and we move on to the second time, the
\textcolor{blue}{key's are already in the cache!}. Hence the time taken for the cache to return the key is lower since it does not have to query the server for the keys
that it does not have.
\begin{equation*}
    t_{client - cache- client} < t_{client-cache-server-cache-client}
\end{equation*}
Requests 2 and 3 are the ones where only the Cache has been queried , and hence the time taken by them is lesser.
